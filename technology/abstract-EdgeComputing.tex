\section{Edge Computing}

Edge computing is a network architecture concept where in the cloud 
computing capabilities are carried out at the edge of cellular network
where the end device or requester is located.

The main idea behind edge computing is to reduce the network latency and
radio network resource consumption by bringing the cloud services closer
to the device so that latency is reduced significantly.

This mechanism requires leveraging or using resources that may not be 
connected to a network with devices such as laptops, smartphones, 
tablets and sensors.

``Edge computing covers a wide range of technologies including wireless 
sensor networks, mobile data acquisition, mobile signature analysis, 
cooperative distributed peer-to-peer ad hoc networking and processing 
also classifiable as local cloud/fog computing and grid/mesh computing, 
dew computing, mobile edge computing,cloudlet, distributed data storage
and retrieval, autonomic self-healing networks,
remote cloud services''~\cite{edge}

Majority of its application are realized in IOT and other smart connected
ecosystem where emergency is the highest priority and data processing
is scarce. Naive example would be a baby crossing a road and an autonous
vehicle running over the same road, needs to decide as soon as possible
to stop motion in order to save baby's life. It cannot send the data to
main cloud server and wait for response which would be time consuming and
baby's life would be at jeopardy.
Hence edge computing would be really useful and saviour for scenarios where
offloading to cloud is considered costly.
