\section{Cloudlet}


A cloudlet is technique or mechanism by which the cloud capabilities
and its wonderful storage,data processing and data analysis power is
brought at the edge of the cellular network.  The main idea behind
cloudlet is to bring the cloud and its services closer to the
client (IOT devices,smart phones, smart watches) which cannot
independently complete the high load of computation and would require
offloadingto meet the computational requirements. This offloading to
the main cloud serverwould take relatively longer time in cases where
the action has to be taken as soonas possible in real time. Thus in
scenarios where latency must be minimum and offloading becomes
compulsory we would then be compelled to use cloudlets, where the
computation now happens at the edge of cellular network and latency is
reduced significantly.``It is a new architectural element that extends
today's cloud computing infrastructure.  It represents the middle tier
of a 3-tier hierarchy: mobile device - cloudlet -
cloud.''~\cite{hid-sp18-410-wikiCloudlet}

Thus a cloudlet can be viewed as a mini data center whose aim is to
bring the cloud closer to the Non powerful devices. ``The cloudlet
term was first coined by Satyanarayanan and a prototype implementation
is developed by Carnegie Mellon University as a research project.The
concept of cloudlet is also known as follow me cloud,and mobile
micro-cloud''~\cite{hid-sp18-410-wikiCloudlet}
